\section{Aufgaben zur \ce{CO2}-Konzentration in Räumen}
\subsubsection*{Wie oft sollte nun gelüftet werden, um einer Verbreitung von Aerosolen vorzubeugen?}

\subsection*{Aufgaben:}
In einem Seminarraum $(V=\SI{200}{\kmeter})$ befinden sich 30 Studenten. Sie führen sitzende Tätigkeiten aus mit einer \ce{CO2}-Abgabe von \SI{15}{\liter \per \hour}. Laut DGUV, der deutschen gesetzlichen Unfallversicherung, soll nach der folgenden Tabelle regelmäßig gelüftet werden. Die \ce{CO2}-Konzentration in der Atmosphäre ist mit 0,04 Vol.\% anzunehmen. Die Luftfeuchtigkeit wird nicht berücksichtigt.

\begin{table}[h!]
	\renewcommand*{\arraystretch}{1.2}
	\centering
	\rowcolors{2}{white}{gray!25}
	\caption{Regelmäßiges Lüften zur Sicherheit vor Corona \cite{e.V..2020}}
	\label{tab:corona_lueften}
	\begin{tabulary}{1.0\textwidth}{C|C|C}
		\hline
		& \textbf{Winter} &\textbf{Sommer}\\
		\textbf{Büroräume}	& in \SI{1}{\hour} für \SI{3}{\minute} lüften&in \SI{1}{\hour} für \SI{10}{\minute} lüften\\
		\textbf{Seminarräume} & in \SI{20}{\minute} für \SI{3}{\minute} lüften & in \SI{20}{\minute} für \SI{10}{\minute} lüften\\
		\hline			
	\end{tabulary}
\end{table}
\FloatBarrier

\begin{enumerate}
	\item [a)] 	Welche Konzentration an \ce{CO2} (ppm) liegt nach 1,5 h Vorlesung vor, wenn der Raum mit einer Luftwechselzahl von n=0,1 kaum gelüftet wird?{\scriptsize  (3530 ppm)}
	
	\item [b)] Nach dem letzten Block wird der Seminarraum gewechselt und einige Studenten sind bereits nach Hause gegangen. Im neuen Seminarraum $\SI{146}{\kmeter}$ befinden sich nun nur noch 21 Studenten. In diesem Raum soll die \textsc{Pettenkofer}-Zahl nicht überschritten werden. Durch das Wechseln des Raumes geben die Studenten für die 45 min Seminar nun \SI{18}{\liter \per \hour}  \ce{CO2 } ab. 
	Wie hoch muss die Luftwechselzahl sein, um den geforderten Grenzwert einzuhalten? {\scriptsize (\SI{3,89}{\per \hour})}
	
	\item [c)] Welche Luftwechselzahl ergibt sich für 15 Studenten im Raum aus 1 a), wenn es Winter ist und die nötige Regelmäßigkeit der Lüftung zur Sicherheit vor Aerosolbildung einzuhalten ist? Reicht dieser Luftwechselzahl um den Grenzwert der DIN 1946-2 oder der Pettenkofer-Zahl einzuhalten?\\
	Es wird davon ausgegangen, dass mit den Fenstern stoßgelüftet wird und entweder alle Fenster und Türen offen oder zu sind.\\
	{\scriptsize (reicht nicht für Pettenkofer oder DIN 1946-2, \SI{0,32}{\per \hour} < \SI{0,62}{\per \hour}< \SI{1,74}{\per \hour})}
	
	\item [d)] Werden die Grenzwerte im Sommer eingehalten? 
	{\scriptsize (reicht nicht für Pettenkofer, aber für DIN 1946-2, \SI{0,62}{\per \hour}< \SI{1,08}{\per \hour} < \SI{1,74}{\per \hour})}
	
	\item [e)] Wie lang muss im Winter und im Sommer für c) und d) pro 20 min gelüftet werden um die Grenzwerte für \textsc{Pettenkofer} und DIN 1946-2 einzuhalten?
	{\scriptsize (Sommer/Winter: DIN mind. alle 20 min 5,8 min lüften, Pettenkofer mind. alle 20 min 16,2 min lüften)}
		
\end{enumerate}