\section{Rechnerisch den \ce{CO2}-Gehalt und die Luftwechselzahl bestimmen}

Um die CO2-Konzentration mit bekannter Luftwechselzahl zu bestimmen ergibt sich mit der Hintergrundbelastung an CO2 aus der Umgebungsluft folgender Term:

\begin{flalign}
	\label{gl:co2}
	c_{\ce{CO2}} (t) &= c_{\ce{CO2}, \text{außen}} + \frac{N*\dot{V}_{\ce{CO2}}}{10*n*V}*\left[1-e^{-n*t}\right]
\end{flalign}

\begin{flalign}
	\label{gl:luftwechsel}
	n &= \frac{N*\dot{V}_{\ce{CO2}}}{10*V*\left[c_{\ce{CO2}}(t\rightarrow \infty)-c_{\ce{CO2},\text{außen}}\right]}
\end{flalign}

\begin{description}
	\item [$ \boldmath c_{\ce{CO2}} (t) \ldots$] Innenraum Konzentration an \ce{CO2} in Vol.\% zu einem Zeitpunkt t
	\item [$\boldmath c_{\ce{CO2},\text{außen}}\ldots$] Außenkonzentration an \ce{CO2} in Vol.\% ($\approx $ 4 Vol.\%)
	\item [$ \boldmath c_{\ce{CO2}} (t\rightarrow \infty)\ldots$] Grenzkonzentration an \ce{CO2} für $t \rightarrow \infty$
	\item[$ \boldsymbol N\ldots$] Anzahl der Personen
	\item[$\boldsymbol n\ldots$] Luftwechselzahl in \si{\per \hour}
	\item[$\boldsymbol V\ldots$] Raumvolumen in \si{\kmeter}
	\item[$\boldmath \dot{V}_{\ce{CO2}}\ldots$] spezifische Emissionsrate in \si{\liter \per \hour}
	\item[$\boldsymbol t\ldots$] Zeit in \si{\hour}
\end{description}

\subsection*{Herleitung der Gleichung für die \ce{CO2}-Konzentration}
\textcolor{red}{Einfacher erklären!!}

\begin{flalign}
	\tag{siehe Gl. \ref{gl:co2}}
	c_{\ce{CO2}, \text{innen}} (t) &= \frac{N*\dot{V}_{\ce{CO2}}}{10*n*V}*\left(1-e^{-n*t}\right) \\
	&= \frac{1}{k_2}*\frac{N*\dot{V}_{\ce{CO2}}}{10*V}*\left(1-e^{-k_2*t}\right)\\
	&= \frac{\frac{\dot{V}_{\ce{CO2}}}{V}}{k_2}*\frac{N}{10}*\left(1-e^{-k_2*t}\right)\\
	&= \frac{k_1}{k_2}*c_{A,0}*\left(1-e^{-k_2*t}\right)\\
	c_b&= \frac{k_1}{k_2}*c_{A,0}*\left(1-e^{-k_2*t}\right)
\end{flalign}