\section{Übung 1}
\subsection*{a$)$}

\begin{minipage}[t]{0.33\textwidth}
$\underline{\text{Gegeben:}}$
\begin{itemize}
	\item V=70 \si{\liter}
	\item p=2 \si{\bar}
	\item $\text{p}_{amb}$=1000 \si{\hecto\pascal}
	\item T=(20+273)\si{\kelvin}=293\si{\kelvin}
\end{itemize}
\end{minipage}
\begin{minipage}[t]{0.33\textwidth}
	$\underline{\text{Gesucht:}}$
	\begin{itemize}
		\item $\rho$
		\item m$_{Luft}$
	\end{itemize}
\end{minipage}
\begin{minipage}[t]{0.33\textwidth}
	$\underline{\text{Verwendete Formeln:}}$
	\begin{equation}
	p*V=n*R*T
	\end{equation}
	\begin{equation}
	\rho =\frac{m}{V}
	\end{equation}
	\begin{equation}
	\text{p}_{abs} =\text{p}+\text{p}_0
	\end{equation}
\end{minipage}

\vspace{1cm}

Der Reifendruckmesser (\textbf{Manometer}) zeigt den Überdruck gegenüber der Atmosphäre an.

\begin{equation}
p_{amb}=\SI{1000}{\hecto\pascal}=\underline{10^5 \si{\pascal}}=\SI{1}{\bar}
\end{equation}
\begin{equation}
	\text{p}_{abs}=10^5\si{\pascal}+2*10^5\si{\pascal}=\underline{3*10^5\si{\pascal}}
\end{equation}
\begin{flalign}
p*V&=n*R*T\\
&=m*R_i*T\\
p&=\frac{m}{V}*R_i*T\\
\rho=\frac{m}{V}&=\frac{p_{abs}}{R_i*T}
\end{flalign}\\
$\Longrightarrow$ spezifische Gaskonstante $R_i$ der Luft aus dem Internetz\\
\begin{flalign}
	\rho&=\frac{3*10^5\si{\pascal}}{\SI{287,6}{\joule\per\kilogram\per\kelvin}*\SI{293}{\kelvin}}\\
	&=\underline{\underline{\SI{3,565}{\kilogram\per\cubic\meter}}}
\end{flalign}
\begin{flalign}
	m&=\rho*V\\
	&=\SI{3,565}{\kilogram\per\cubic\meter}*\SI{0,07}{\cubic\meter}\\
	&=\underline{\underline{\SI{0,25}{\kilogram}}}
\end{flalign}





\subsection*{b$)$}
\begin{minipage}[t]{0.33\textwidth}
	$\underline{\text{Gegeben:}}$
	\begin{itemize}
		\item $V_1=\SI{0,5}{\liter}$
		\item $p_{1,ü}=\SI{1,7}{\bar}$
		\item $V_2=\frac13 *V_1 $
		\item $p_{amb}=\SI{1,013}{\bar}$
		\item $T=\SI{293}{\kelvin}$ (isotherm)
	\end{itemize}
\end{minipage}
\begin{minipage}[t]{0.33\textwidth}
	$\underline{\text{Gesucht:}}$
	\begin{itemize}
		\item $p_2$
		\item Überdruck oder Absolutdruck
	\end{itemize}
\end{minipage}
\begin{minipage}[t]{0.33\textwidth}
	$\underline{\text{Verwendete Formeln:}}$
	\begin{equation}
	\text{p}_{abs} =p_{ü}+p_{amb}
	\end{equation}
	\begin{equation}
p_1*V_1=p_2*V_2	
	\end{equation}
	
\end{minipage}

\vspace{1cm}

Das \textbf{Manometer} zeigt den Überdruck gegenüber der Atmosphäre an.

\begin{equation}
\text{p}_{1,abs}=1,013*10^5\si{\pascal}+1,7*10^5\si{\pascal}=\underline{2,713*10^5\si{\pascal}}
\end{equation}

\begin{flalign}
	p_{1,abs}*V_1&=p_{2,abs}*V_2\\
	p_{2,abs}	&=\frac{p_1*V_1}{V_2}=\frac{2,713*10^5\si{\pascal}*\SI{0,5}{\liter}}{\frac13 *\SI{0,5}{\liter}}\\
	&=\underline{8,139*10^5\si{\pascal}}
\end{flalign}
\begin{flalign}
	p_{2,ü}&=p_{2,abs}-p_{amb}=8,139*10^5\si{\pascal}-1,013*10^5\si{\pascal}\\
	&=\underline{7,126*10^5\si{\pascal}}=\underline{\underline{7,126\si{\bar}}}
\end{flalign}

\subsection*{c$)$}
\begin{minipage}[t]{0.33\textwidth}
	$\underline{\text{Gegeben:}}$
	\begin{itemize}
		\item $\vartheta=\SI{25}{\degreeCelsius} $
		\item $\rho= \SI{1,5}{\kilogram\per\cubic\meter}$
		\item $p_{amb}=\SI{97000}{\pascal}$ 
	\end{itemize}
\end{minipage}
\begin{minipage}[t]{0.33\textwidth}
	$\underline{\text{Gesucht:}}$
	\begin{itemize}
		\item $p_{ü}$ im Behälter
	\end{itemize}
\end{minipage}
\begin{minipage}[t]{0.33\textwidth}
	$\underline{\text{Verwendete Formeln:}}$
	\begin{equation}
	p*V=n*R*T
	\end{equation}
	\begin{equation}
p_{abs} =p_{ü}+p_{amb}
	\end{equation}
	
\end{minipage}
\begin{flalign}
	p*V&=n*R*T\\
	p*V&=m*R_i*T\\
	\rho=\frac{m}{V}&=\frac{p}{R_i*T}\\
	\SI{1,5}{\kilogram\per\cubic\meter}&=\frac{p_{abs}}{\SI{296,8}{\joule\per\kilogram\per\kelvin}*\SI{298,15}{\kelvin}}\\
	p_{abs}&=\underline{\SI{132736}{\pascal}}\\
	p_{ü}&=p_{abs}-p{amb}=\SI{132736}{\pascal}-\SI{97000}{\pascal}\\
	p_{ü}&=\underline{\underline{\SI{35736}{\pascal}=\SI{0,357}{bar}}}
\end{flalign}

\subsection*{d$)$}
\begin{minipage}[t]{0.33\textwidth}
	$\underline{\text{Gegeben:}}$
	\begin{itemize}
		\item Kugeltank
		\item $\vartheta=\SI{10}{\degreeCelsius}$
		\item $p_{1,Mano}=\SI{100}{\kilo\pascal}$
		\item $p_{2,Mano}=\SI{200}{\kilo\pascal}$
		\item $p_{amb}=\SI{100}{\kilo\pascal}$
	\end{itemize}
\end{minipage}
\begin{minipage}[t]{0.33\textwidth}
	$\underline{\text{Gesucht:}}$
	\begin{itemize}
		\item $\frac{m_{1,gas}}{m_{2,gas}}$
		
	\end{itemize}
\end{minipage}\begin{minipage}[t]{0.33\textwidth}
	$\underline{\text{Verwendete Formeln:}}$
	\begin{equation}
	p_1*V_1=p_2*V_2
	\end{equation}
	\begin{equation}
	p_{abs} = p_{ü}+p_{amb}
	\end{equation}
	\begin{equation}
	p*V=M*R_i*T
	\end{equation}
\end{minipage}

\begin{flalign}
	p_{abs}&=p_{amb}+p_{ü}=p_{amb}+p_{Mano,1,2}\\
	p_1&=\SI{100}{\kilo\pascal}+\SI{100}{\kilo\pascal}=\SI{200}{\kilo\pascal}\\
	p_2&=\SI{100}{\kilo\pascal}+\SI{200}{\kilo\pascal}=\SI{300}{\kilo\pascal}
\end{flalign}

Aus dem Idealen Gasgesetz folgt, dass die Masse proportional dem Volumen ist. Daher:
\begin{flalign}
	V_1*p_1&=V_2*p_2\\
	m_1*p_1&=m_2*p_2\\
	\frac{m_1}{m_2}&=\frac{p_1}{p_2}\\
	&=\frac{\SI{200}{\kilo\pascal}}{\SI{300}{\kilo\pascal}}\\
	&=\underline{\underline{\frac23}}
\end{flalign}