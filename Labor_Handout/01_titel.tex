\begin{center}
\begin{tabular}{p{\textwidth}}


\begin{center}
\includegraphics[scale=0.75]{logos.jpg}\\
\end{center}


\\

\begin{center}
\LARGE{\textsc{Einführung in die Laborpraktika\\
}}
\end{center}

\\

%\begin{center}
%\large{Fakultät für Muster und Beispiele \\
%der Hochschule Musterhausen \\}
%\end{center}
%
%\\

\begin{center}
\textbf{\Large{Handout mit allgemeinen Hinweisen für \mbox{chemie- und umwelttechnische} Praktika}}
\end{center}


\\
%\begin{center}
%zur Erlangung des akademischen Grades\\
%Bachelor of Engineering
%\end{center}


%\begin{center}
%vorgelegt von
%\end{center}

\begin{center}
	\includegraphics[scale=0.75]{img/versuchsaufbau_1}\\
\end{center}
%Ende

\begin{center}
	Diese Übersicht soll für zukünftige Praktika eine Unterstützung bieten, um Geräte oder Versuchsstände selbstständig aufbauen und bedienen zu können.
\end{center}


\\ \\

%\begin{center}
%\begin{tabular}{lll}
%\large{\textbf{Protokollführer:}} & & \large{Roman-Luca Zank}\\
%&&\\
%\large{\textbf{Datum der Versuchsdurchführung:}}&& \large{22.10.2020}\\
%&&\\
%\large{\textbf{Abgabedatum:}}&& \large{\todayDE}
%\end{tabular}
%\end{center}

\end{tabular}
\end{center}

\vfill
\large{Merseburg den \todayDE}
