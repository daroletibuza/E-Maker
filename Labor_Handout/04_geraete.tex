\section{Laborgeräte und Werkzeuge}
Im Umgang mit Laborgeräten ergeben sich mehrere Fehlerquellen, welche in der Auswertung von Versuchen relevant sein können. Zu dem sollte jeweils der Nutzen des jeweiligen Arbeitsmittels bekannt sein, um Messungenauigkeiten zu vermeiden.

\subsection{Allgemeiner Apparaturaufbau}
\subsubsection{Klammern}
\subsubsection{Muffen}
\subsubsection{Stative}
\subsubsection{Korkringe}

\subsection{Volumengefäß}
\subsubsection{Bechergläser}
\subsubsection{Rundkolben}
\subsubsection{Erlenmeyerkolben}
\subsubsection{Maßkolben}
\subsubsection{Bürette}

\subsection{Pipetten}
\subsubsection{Peleusball}
\subsubsection{Vollpipetten}
\subsubsection{Eppendorfpipetten}
\subsubsection{Hubkolbenpipette}

\subsection{Trichter}
\subsubsection{Flüssigkeitstrichter}
\subsubsection{Feststofftrichter}
\subsubsection{Tropftrichter}
\subsubsection{Scheidetrichter}

\subsection{Schläuche}
\subsubsection{Vakuumschläuche}
\subsubsection{Wasserschläuche}
\subsubsection{Oliven}

\subsection{Filter}
\subsubsection{Filterpapier}
\subsubsection{Fritte}
\subsubsection{Filternutsche}

\subsection{Waschflaschen}

\subsection{Rührer}
\subsubsection{Magnetrührwerk}
\subsubsection{Rührertypen}
\subsubsection{Rührermotor}

\subsection{Rückflusskühler}
\subsubsection{Dimrothkühler}
\subsubsection{Liebigkühler}

\subsection{Heizelemente}
\subsubsection{Wärmebad}
\subsubsection{Brenner}
\subsubsection{Heizpilz oder Heiznetz}
\subsubsection{Heizplatte}

\subsection{Pyknometer}

\subsubsection{Apparaturen zum Trocknen}
\subsubsection{Exsikkator}
\subsubsection{Trockenschrank}
\subsubsection{Muffelofen}

\subsection{Pumpen}
\subsubsection{Vakuumpumpe (Wasserstrahlpumpe)}
\subsubsection{Hubkolbenpumpe}
\subsubsection{Kreiselpumpe}

\subsection{Zusätzlich:}
\subsubsection{Beschriftung von Proben}
\subsection{Füllkörper}
\subsubsection{Schliffe und Schlifffett}
\subsubsection{Ultraschallbad}
\subsubsection{Eismaschine}