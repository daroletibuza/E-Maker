\section{Typische Versuchsstände und Aufgaben}

\subsection{Typische Versuchsstände}
\subsubsection*{Becherglas-Rührapparatur}
\subsubsection*{Rückflussapparatur}
\subsubsection*{Mehrhalskolbenapparatur}
\subsubsection*{Titrationsapparatur}

\subsection{Typische Verfahren und Aufgabenstellungen}
\subsubsection*{Dichtebestimmung}
\subsubsection*{Trocknung von Feststoffen}
\subsubsection*{Destillation}
\subsubsection*{Umkristallisieren}
http://www.stalke.chemie.uni-goettingen.de/virtuelles_labor/basics/9_more_de.html
\subsubsection*{Extraktion}
\label{sec:extraktion}
Auschütteln erklären
Tipp mit gesättigter NaCL Lösung bei Organik
\subsubsection*{Absaugen alias Vakuumfiltrieren}
\subsubsection*{Schmelzpunkt}
\subsubsection*{Siedepunkt}
\subsubsection*{Refraktometrie}
\subsubsection*{Dünnschichtchromatographie}

\subsubsection*{Gaschromatografie}
\subsubsection*{Wasser entzug von organischen Lösungen mit Na2SO4 oder CaCl2}