\section{Sicherheitsmaßnahmen}
\begin{itemize}
	\item es ist sich stets über den durchzuführenden Versuchsaufbau, sowie die genutzten Stoffe/Chemikalien zu informieren und deren Gefahr abzuschätzen
	\item das Tragen von Schutzkleidung ist Pflicht (Kittel, Brille, evtl. Handschuhe)
	\item je nach Risiko sind die Versuche nur unter Beaufsichtigung oder unter einem Abzug durchzuführen
	\item  Essen und Trinken ist im Labor verboten
	\item  Hände sollten nach der Versuchsdurchführung gewaschen werden 
	\item Um an höhere liegende Objekte zu gelangen, ist eine Leiter (zu zweit) oder ein Elefantenfuß zu nutzen
	\item Fluchtwege sind stets freizuhalten
	\item \textbf{Notfalltelefonnummern:}
		\begin{itemize}
			\item \textbf{Labortelefon: 2666}
			\item \textbf{Handy: 112}
		\end{itemize}
	\item \textbf{{\Large Hilfe holen !}}
\end{itemize}
