\section{Übung 2}
\subsection*{a$)$}

\begin{minipage}[t]{0.33\textwidth}
$\underline{\text{Gegeben:}}$
\begin{itemize}
	\item $F_{Hand}=\SI{100}{\newton}$
	\item $d_1=\SI{0,015}{\meter}$
	\item $d_2=\SI{0,05}{\meter}$
	\item $l_1=\SI{0,33}{\meter}$
	\item $l_2=\SI{0,03}{\meter}$
\end{itemize}
\end{minipage}
\begin{minipage}[t]{0.33\textwidth}
	$\underline{\text{Gesucht:}}$
	\begin{itemize}
		\item Übersetzungsverhältnis gesamt $\frac{F_2}{F_{Hand}}$
		\item Übersetzung hydrau\-lische Seite
	\end{itemize}
\end{minipage}
\begin{minipage}[t]{0.33\textwidth}
	$\underline{\text{Verwendete Formeln:}}$
	\begin{equation}
	\frac{F_1}{F_2}=\frac{A_2}{A_1}=\frac{l_2}{l_1}
	\end{equation}
\end{minipage}

\vspace{1cm}

Beachte, es liegt ein einseitiger Hebel vor! Darum besteht der lange Hebelarm $l_1$ aus dem gesamten Hebel, und nicht nur aus einer Seite vom Gelenk aus.
\begin{flalign}
A1&= \frac{\pi}{4}*d_1^2= \frac{\pi}{4}*(\SI{0,015}{\meter})^2\\
&= \underline{\SI{176,5e-6}{\smeter}}\\
A2&= \frac{\pi}{4}*d_2^2= \frac{\pi}{4}*(\SI{0,05}{\meter})^2\\
&= \underline{\SI{1962,5e-6}{\smeter}}
\end{flalign}

Wirkung des Hebels:
\begin{flalign}
	\frac{F_{Hand}}{F1}&=\frac{l_1}{l_2}\\
	F_1&=F_{Hand}*\frac{l_1}{l_2}=\SI{100}{\newton}*\frac{\SI{0,33}{\meter}}{\SI{0,03}{\meter}}\\
	&=\underline{\SI{1100}{\newton}}
\end{flalign}
Wirkung der Hydraulik:
\begin{flalign}
	\frac{F_1}{F_2}&=\frac{A_2}{A_1}\\
	F_2&=F_1*\frac{A_2}{A_1}\\
	&=\SI{1100}{\newton}*\frac{\SI{1962,5e-6}{\smeter}}{\SI{176,5e-6}{\smeter}}\\
	&=\underline{\SI{12230,9}{\newton}}
\end{flalign}
\begin{flalign}
	\frac{F_{2}}{F_{Hand}}&=\frac{\SI{12230,9}{\newton}}{\SI{100}{\newton}}=\underline{\underline{122,309}}
\end{flalign}
\begin{flalign}
\frac{F_{2}}{F_{1}}&=\frac{\SI{12230,9}{\newton}}{\SI{1100}{\newton}}=\underline{\underline{11,119}}
\end{flalign}


\subsection*{b$)$ - normal}

\begin{minipage}[t]{0.33\textwidth}
	$\underline{\text{Gegeben:}}$
	\begin{itemize}
		\item $F=\SI{200}{\newton}$
		\item $d_1=\SI{0,03}{\meter}$
		\item $d_2=\SI{0,02}{\meter}$
		\item $l_1=\SI{0,35}{\meter}$
		\item $l_2=\SI{0,07}{\meter}$
	\end{itemize}
\end{minipage}
\begin{minipage}[t]{0.33\textwidth}
	$\underline{\text{Gesucht:}}$
	\begin{itemize}
		\item Systemdruck $p$
		\item Anpressdruck der Bremsbacken $F_2$
	\end{itemize}
\end{minipage}
\begin{minipage}[t]{0.33\textwidth}
	$\underline{\text{Verwendete Formeln:}}$
	\begin{equation}
	\frac{F_1}{F_2}=\frac{A_2}{A_1}=\frac{l_2}{l_1}
	\end{equation}
	\begin{equation}
		p=\frac{F}{A}
	\end{equation}
\end{minipage}

\vspace{1cm}

Wirkung des Hebels:
\begin{flalign}
\frac{F}{F_1}&=\frac{l_1}{l_2}\\
F_1&=F*\frac{l_1}{l_2}=\SI{200}{\newton}*\frac{\SI{0,35}{\meter}}{\SI{0,07}{\meter}}\\
&=\underline{\SI{1000}{\newton}}
\end{flalign}


Druck im System:

Druck ist Kraft pro Fläche.

\begin{flalign}
	p_1&=\frac{F1}{A1}\\
	&=\frac{\SI{1000}{\newton}}{\frac{\pi}{4}*(\SI{0,03}{\meter})^2}\\
	&=\underline{\underline{\SI{1,415E6}{\pascal}}}
\end{flalign}

Druck auf die Bremsbacken:

\begin{flalign}
F&=A*p\\
F_2&=A_2*p_1=\left( \frac{\pi}{4}*(\SI{0,02}{\meter})^2 \right) *\SI{1,415E6}{\pascal}\\
&= \underline{\underline{\SI{444,5}{\newton}}}
\end{flalign}



\newpage

\subsection*{b$)$ - Druckwandler}

\begin{minipage}[t]{0.33\textwidth}
	$\underline{\text{Gegeben:}}$
	\begin{itemize}
		\item $d_1=\SI{0,016}{\meter}$
		\item $d_{1,2}=\SI{0,04}{\meter}$
		\item $d_2= \SI{0,018}{meter}$
		\item $l_1=\SI{0,35}{\meter}$
		\item $l_2=\SI{0,07}{\meter}$
	\end{itemize}
\end{minipage}
\begin{minipage}[t]{0.33\textwidth}
	$\underline{\text{Gesucht:}}$
	\begin{itemize}
		\item Benötigte Kraft am Bremspedal ($F_{neu}$)
		
	\end{itemize}
\end{minipage}
\begin{minipage}[t]{0.33\textwidth}
	$\underline{\text{Verwendete Formeln:}}$
	\begin{equation}
	\frac{F_1}{F_2}=\frac{A_2}{A_1}=\frac{l_2}{l_1}
	\end{equation}
	\begin{equation}
		\frac{p_1}{p_2}=\frac{A_2}{A_1}=\frac{d_2}{d_1}
	\end{equation}
\end{minipage}

\vspace{1cm}

\begin{flalign}
	\frac{F_{neu}}{F_1}&=\frac{l_2}{l_1}\\
	F_1&=p_2*A_2\\
	p_2&=\frac{d_1^2*p_1}{d_{1,2}^2}
\end{flalign}
Zusammengesetzt:
\begin{equation}
	\frac{l_2}{l_1}=\frac{F_{neu}}{\left( \frac{d_1^2*p_1}{d_{1,2}^2} \right)  *A_2}
\end{equation}
\begin{flalign}
F_{neu}&=\frac{A_2*d_1^2*p_1*l_2}{d_{1,2}^2*l_1}\\
&=\frac{(\frac{\pi}{4}*(d_2)^2)*d_1^2*p_1*l_2}{d_{1,2}^2*l_1}\\
&=\frac{(\frac{\pi}{4}*(\SI{0,018}{meter})^2)*(\SI{0,016}{\meter})^2*\SI{1,415E6}{\pascal}*\SI{0,07}{\meter}}{(\SI{0,04}{\meter})^2*\SI{0,35}{\meter}}\\
&=\underline{\underline{\SI{11,52}{\newton}}}
\end{flalign}